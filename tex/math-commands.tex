% LaTeX math commands

% Units
\newcommand{\uni}[1]{\ \mathrm{#1}}

\newcommand{\sciuni}[2]{\cdot10^{#1} \ \mathrm{#2}}

\newcommand{\dgr}{^{\circ}}

% Symbols, variables and operators
\newcommand{\ham}{\mathcal{H}}
\newcommand{\lag}{\mathcal{L}}


% Hyperbolic functions (only sech and csch as the others are defined by default)
\DeclareMathOperator{\sech}{sech}
\DeclareMathOperator{\csch}{csch}

% Inverse hyperbolic functions
\DeclareMathOperator{\arsinh}{arsinh}
\DeclareMathOperator{\arcosh}{arcosh}
\DeclareMathOperator{\artanh}{artanh}
\DeclareMathOperator{\arsech}{arsech}
\DeclareMathOperator{\arcsch}{arcsch}
\DeclareMathOperator{\arcoth}{arcoth}

% Inverse trigonometric functions (only arcsec, arccsc and arccot as the others are defined by default)
\DeclareMathOperator{\arcsec}{arcsec}
\DeclareMathOperator{\arcsc}{arccsc}
\DeclareMathOperator{\arccot}{arccot}

% Functions
\DeclareMathOperator{\erf}{erf}

% Complex analysis
\DeclareMathOperator*{\res}{Res}

\DeclareMathOperator{\wind}{wind}

\DeclareMathOperator{\re}{Re}

\DeclareMathOperator{\im}{Im}

\DeclareMathOperator{\Arg}{Arg}

\DeclareMathOperator{\Ln}{Ln}

\DeclareMathOperator{\pv}{P.V.}

% Linear algebra

% Defined by default: det, dim, ker

\DeclareMathOperator{\tr}{tr}

\DeclareMathOperator{\rank}{rank}

\newcommand{\inprod}[2]{\langle#1,#2\rangle}

% ------------------------------------------

\newcommand{\bra}[1]{\big\langle#1\big\rvert}
\newcommand{\Bra}[1]{\Big\langle#1\Big\rvert}
\newcommand{\sbra}[1]{\langle#1\rvert}

% ------------------------------------------

\newcommand{\ket}[1]{\big\lvert#1\big\rangle}
\newcommand{\Ket}[1]{\Big\lvert#1\Big\rangle}
\newcommand{\sket}[1]{\lvert#1\rangle}

% ------------------------------------------

\newcommand{\braket}[2]{\big\langle#1\big\lvert#2\big\rangle}
\newcommand{\Braket}[2]{\Big\langle#1\Big\lvert#2\Big\rangle}

% ------------------------------------------

\newcommand{\ketbra}[2]{\big\lvert#1\big\rangle\big\langle#2\big\rvert}
\newcommand{\Ketbra}[2]{\Big\lvert#1\Big\rangle\Big\langle#2\Big\rvert}

% ------------------------------------------

\newcommand{\exval}[1]{\big\langle#1\big\rangle}
\newcommand{\Exval}[1]{\Big\langle#1\Big\rangle}

% ------------------------------------------

\newcommand{\exvstate}[2]{\big\langle#2\big\lvert#1\big\rvert#2\big\rangle}
\newcommand{\Exvstate}[2]{\Big\langle#2\Big\lvert#1\Big\rvert#2\Big\rangle}

% ------------------------------------------

\newcommand{\matel}[3]{\big\langle#1\big\lvert#2\big\rvert#3\big\rangle}
\newcommand{\Matel}[3]{\Big\langle#1\Big\lvert#2\Big\rvert#3\Big\rangle}

% ------------------------------------------

\newcommand{\comm}[2]{\big[#1, #2\big]}
\newcommand{\Comm}[2]{\Big[#1, #2\Big]}

% ------------------------------------------

\newcommand{\acomm}[2]{\big\{#1, #2\big\}}
\newcommand{\Acomm}[2]{\Big\{#1, #2\Big\}}

% Vectors

\newcommand{\unitv}[1]{\mathbf{\hat{#1}}}

\newcommand{\unitg}[1]{\boldsymbol{\hat{#1}}}

\newcommand{\vtr}[1]{\mathbf{#1}}

\newcommand{\vtg}[1]{\boldsymbol{#1}}


% Calculus

\newcommand{\eval}[3]{\left[#1\right]_{#2}^{#3}}

\newcommand{\Eval}[3]{\Big[#1\Big]_{#2}^{#3}}

\newcommand{\partc}[3]{\left(\pdv{#1}{#2}\right)_{#3}}

\newcommand{\totdf}[3]{\left(\pdv{#1}{#2}\right)_{#3}\odif{#2} + \left(\pdv{#1}{#3}\right)_{#2}\odif{#3}}

\newcommand{\ft}[2]{\mathcal{F}[#1](#2)}

\newcommand{\ift}[2]{\mathcal{F}^{-1}[#1](#2)}

\newcommand{\lt}[2]{\mathcal{L}[#1](#2)}

\newcommand{\ilt}[2]{\mathcal{L}^{-1}[#1](#2)}

% Sets

\newcommand{\set}[1]{\mathbb{#1}}

% Combinatorics

\newcommand{\risingfact}[1]{^{\overline{#1}}}

\newcommand{\fallingfact}[1]{^{\underline{#1}}}
